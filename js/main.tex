% Created 2024-03-08 Fri 18:36
% Intended LaTeX compiler: pdflatex
\documentclass[11pt]{article}
\usepackage[utf8]{inputenc}
\usepackage[T1]{fontenc}
\usepackage{graphicx}
\usepackage{longtable}
\usepackage{wrapfig}
\usepackage{rotating}
\usepackage[normalem]{ulem}
\usepackage{amsmath}
\usepackage{amssymb}
\usepackage{capt-of}
\usepackage{hyperref}
\usepackage{sectsty}
\sectionfont{\centering}
\author{Manjot dev}
\date{\today}
\title{JavaScript by}
\hypersetup{
 pdfauthor={Manjot dev},
 pdftitle={JavaScript by},
 pdfkeywords={},
 pdfsubject={},
 pdfcreator={Emacs 29.2 (Org mode 9.7)}, 
 pdflang={English}}
\begin{document}

\maketitle
\section*{Introduction to Programing}
\label{sec:orgaf2f964}
Programing is a  way to talk to computers. A language like  English can be use to  talk to a human but for computers  we need straight forword instructions
\subsection*{Computer is Dumb!}
\label{sec:orga299c87}
When was the last time you ordered some cereal and got DVDs of serial ? \\
Programing is the act of constructing a program, a set of precise instructions telling a computer what to do.
\subsection*{What is Ecmascript?}
\label{sec:org9e9b420}
ECMA Scipt is a standard on which javascript is based!
It was created to ensure that different documents on javascript are actually talking about the some language. \\
JavaScript \& ECMA script can almost always be used interchangably javascript  is very liberal  in what it allows.
\subsection*{How to execute JavaScript?}
\label{sec:org769762b}
JavaScript can be executed right inside one's browser. You can open the javascript console and start writing javascript there. \\
Another way to execute javascript is a runtime like Node.js which can be installed and used to run javascript code. \\
Yet another way to execute javascript is by inserting it inside <script> tag of an HTML documents

\newpage
\section*{Chapter 1 Variables \& Data}
\label{sec:orgd0fb8f4}
Just like we follow some rules white speaking english(the grammar) we have some rules to follow while writing a javascript program. The set of these rules is called syntax in javascript
\subsection*{What is a varible ?}
\label{sec:orge988bd8}
A variable is a container that stores a value.This is very similar to the containers used to store rice water and cats (treat this as an analagy!) \\
This value of a javascript variable can be changed during the execution of a program

var a = 7;

let a = 7;
\subsection*{Rules for choosing variable names}
\label{sec:org03ea81d}
\begin{itemize}
\item Letters, digits, underscores \& \$ sign allowed
\item Must begin with a \$, \_ or a letter
\item JavaScript reserved words cannot be used as a variable name
\item Harry \& harry are different variable (case sensitive)
\end{itemize}
\subsection*{Var vs Let in javaScript}
\label{sec:org06c4922}
\begin{itemize}
\item var is globally scoped while let \& const are block scoped
\item var can be updated \& re-declared within its scope
\item let can be updated but not re-declared
\item const can neither be updated nor be re-declared
\item var variable are initialjed with undefind where as let and const variable are not initialjed
\item const must be initialjed during declaration unlike let and var
\end{itemize}
\subsection*{Prinitive data type \& objects}
\label{sec:org27ecfec}
Primitive data type are a set of basic data types in javascript \\
Object is a non prinitive datatype in javascript.
objects is a non prinitive datatype in javascript
these are the 7 prinitive datatype in javascript

\begin{itemize}
\item Null
\item Number
\item String
\item Symbol
\item Undefined
\item Boolean
\item Biglnt
\end{itemize}
\subsection*{Object}
\label{sec:orgdcd6c8c}
An object in JavaScript can be created as follows.\\

\begin{verbatim}
const item = {
    name: "Led Bulb",
    price: "150"
};
\end{verbatim}
\subsection*{Quick Quiz}
\label{sec:orgdf37bd4}
Write a javascript program to store name, phone number and marks of a student using objects.

\newpage
\section*{Chapter 2 Expressions \& conditionals}
\label{sec:orgc2490d1}
A fragment of code that produces a value is called an expression. Every value written literally is an expression for ex: 77 or ``Manjot''
\subsection*{Operators in JavaScript}
\label{sec:org74cf058}
\begin{enumerate}
\item Arithmetic Operators \\
\begin{verbatim}
        +  = Addition
        -  = Subtraction
        *  = Multiplication
        ** = Exponentiation
        /  = Division
        %  = Modulus
        ++ = Increment
        -- = Decerment
\end{verbatim}

\item Assignment Operators
\begin{verbatim}
        =    x=y
        +=   x=x+y
        -=   x=x-y
        *=   x=x*y
        /=   x=x/y
        %=   x=x%y
        **=  x=x**y
\end{verbatim}

\item Comparison Operators
\begin{verbatim}
        ==   equal to
        !=   not equal
        ===  equal value and type
        !==  not equal value or not equal type
        >    greater than
        <    less than
        >=   greater than or equal to
        <=   less than or equal to
        ?    ternary operator
\end{verbatim}
\item Logical Operators
\begin{verbatim}
        &&   logical and
        ||   logical or
        !    logical not
\end{verbatim}
\end{enumerate}
\end{document}
